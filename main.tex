\documentclass[11pt,a4paper]{article}
\usepackage{authblk}

\usepackage{epsfig}

\usepackage{url}

\usepackage{color}

\usepackage{tikz}
\usetikzlibrary{decorations.pathreplacing,angles,quotes}

\usepackage{natbib}
\usepackage{mathtools}

\newcommand{\defeq}{\vcentcolon=}



\usepackage{hyperref}
\hypersetup{
    colorlinks = true,
    urlcolor   = blue,
    citecolor  = black,
}

\title{Shear effects on two tandem wind turbines' performance using the actuator line method}
%
\author[1]{Abdelmalek Bouaziz}
\author[1]{Radouan Boukharfane\thanks{radouan.boukharfane@um6p.ma}}
\affil[1]{College of Computing, Mohammed VI Polytechnic University (UM6P), Benguerir, Morocco}
\date{}

\begin{document}

\maketitle

\section{Numerical method}

The governing equations for fluid dynamics are the Navier–Stokes equations for incompressible flow, augmented by a subgrid-scale (SGS) model to account for the unresolved scales. In the large-eddy simulation (LES) framework, a filtering operator, denoted as $\widetilde{\bullet}$, is applied to project a field onto the LES grid. 
%
Using this notation, the evolution equations for the LES-resolved velocity field $\widetilde{\boldsymbol{u}}$ are expressed as:
%
%
\begin{align}
&\nabla \cdot \widetilde{\boldsymbol{u}} = 0, \label{eq:mass_conservation} \\
&\frac{\partial \widetilde{\boldsymbol{u}}}{\partial t} + \widetilde{\boldsymbol{u}} \cdot \nabla \widetilde{\boldsymbol{u}} = 
-\frac{1}{\rho} \nabla \widetilde{p} + \nu \nabla^2 \widetilde{\boldsymbol{u}} 
+ \nabla \cdot \widetilde{\boldsymbol{\tau}}^{\texttt{sgs}} + \boldsymbol{f}, \label{eq:momentum_conservation}
\end{align}
%
where $\rho$, $\nu$, and $p$ represent the fluid density, kinematic viscosity, and pressure field respectively. The term $\boldsymbol{f}$ denotes the body force, while $\widetilde{\tau}_{ij}^{\texttt{sgs}}=\widetilde{u_iu_j}-\widetilde{u_i}\widetilde{u_j}$ corresponds to the SGS stress tensor.
%
In this work, only the classical Boussinesq hypothesis \citep{boussinesq1877essai} is used for the models, which suggests that the sub-grid stress tensor can be formulated like the viscous stress tensor by using a turbulent viscosity $\nu_t$ such that:
\begin{equation}
\widetilde{\tau}_{ij}^{\texttt{sgs}}=\nu_t\left(\frac{\partial\widetilde{u}_i}{\partial x_j}-\frac{\partial\widetilde{u}_j}{\partial x_i}\right)-\frac23\nu_t\frac{\partial\widetilde{u}_k}{\partial x_k}\delta_{ij}
\end{equation}
%
The retained model for $\nu_t$ is the model of \citet{germano1991dynamic} given by 
\begin{equation}
\nu_t = \left(\mathcal{C}^d_s\Delta\right)^2\sqrt{2\widetilde{\mathcal{S}}_{ij}\widetilde{\mathcal{S}}_{ij}}
\end{equation}
where $\widetilde{\boldsymbol{\mathcal{S}}}$ represents the strain rate of the resolved structures, $\Delta$ represents the characteristic filter and the constant $\mathcal{C}^d_s$ is based on the Germano identity \citep{germano1991dynamic} and follows the Lilly procesdure \citep{lilly4proposed}.
%
%
In the context of the actuator line method, the body force $\boldsymbol{f}$ is used to model the effect of wind turbines on the flow field \citep{sorensen2002numerical, mikkelsen2003actuator, troldborg2009actuator}. This method introduces body forces distributed along rotating lines to represent the aerodynamic loads exerted by the wind turbine blades. By employing this approach, the complexities of handling a moving mesh for the blades are avoided. 
%

In the actuator line framework, the wind turbine blades are discretized into elements along the blade span, each with a prescribed rotational motion. The body forces are derived from the two-dimensional force per unit span, $\boldsymbol{\text{F}}_{\texttt{2D}}$, expressed as:
%
\begin{equation}
\boldsymbol{\text{F}}_{\texttt{2D}} = (\text{F}_l, \text{F}_d) = 
\left( 
\frac{1}{2} \rho u_r^2 c C_l, 
\frac{1}{2} \rho u_r^2 c C_d 
\right),
\label{eq:2d_forces}
\end{equation}
%
where $\text{F}_l$ and $\text{F}_d$ are the lift and drag forces, respectively. The lift force $\text{F}_l$ acts perpendicular to the relative velocity vector, while the drag force $\text{F}_d$ acts parallel to it (cf. Fig.~\ref{fig:2dref}). 
%
The relative velocity $u_r$ is computed as:
\begin{equation}
u_r = \sqrt{u_y^2 + u_z^2},
\end{equation}
%
where $u_y$ and $u_z$ are the velocity components in the respective directions. The terms $c$, $C_l$, and $C_d$ represent the local chord length, the two-dimensional lift coefficient, and the drag coefficient, respectively. The coefficients $C_l$ and $C_d$ are obtained from airfoil data based on the local Reynolds number and angle of attack $\alpha$, with $\alpha$ calculated from the local relative velocity.
%
\begin{figure}
\centering
\begin{tikzpicture}
  \def\c{7.5}
  \def\r{0.35*\c}
  % Some profiles look better when using plot[smooth]
  \draw[scale=\c,rotate=-10,fill=black!10] plot file{figs/naca4415.dat} -- cycle;
  \draw[rotate=-10,dashed] (-0.55*\c,0)--(0.85*\c,0);
  \draw[->, >=stealth,olive] (-0.65*\c,0)--(0.85*\c,0) node[below,black]{\Large $z$};
  \draw[->, >=stealth,olive] (0,-0.2*\c)--(0,0.55*\c) node[left,black]{\Large $y$};
  \draw[->, >=stealth,gray] (0,0)--(0.7*\c,0) node[below]{\Large $u_z$};
  \draw[->, >=stealth,gray] (0,0)--(0,0.2*\c) node[right]{\Large $u_y$};
  \draw[->, >=stealth,gray] (0,0)--(0.7*\c,0.2*\c) node[right]{\Large $u_{r}$};
  \draw[dashed] (0,0)--(-0.7*0.75*\c,-0.2*0.75*\c);
  \draw[->, >=stealth,very thick] (0,0)--(0.7*0.12*\c,0.2*0.12*\c) node[above]{\Large $\text{F}_{d}$};
  \draw[->, >=stealth,very thick] (0,0)--(-0.2*0.7*\c,0.7*0.7*\c) node[left]{\Large $\text{F}_{l}$};
  \draw[-stealth, blue] (0,0) +(180:\r) arc(180:170:\r) node[left,midway]{\Large $\alpha_g$};
  \draw[-stealth, blue] (0,0) +(195.945:1.4*\r) arc(195.945:170:1.4*\r) node[left,midway]{\Large $\alpha$};
\end{tikzpicture}


\caption{Local reference frame defined by the blade's cross-sectional geometry.}
\label{fig:2dref}
\end{figure}
%
To address the dimensional inconsistency between the two-dimensional force per unit length, $\boldsymbol{\text{F}}_{\texttt{2D}}$, and the body force, $\boldsymbol{f}$, an intermediate idealized three-dimensional body force, denoted as $\boldsymbol{f}^i$, is introduced. This intermediate force is derived from $\boldsymbol{\text{F}}_{\texttt{2D}}$ and is expressed as:
%
\begin{equation}
\label{eq:f2d}
\boldsymbol{f}^i = \frac{1}{\rho} \boldsymbol{\text{F}}_{\texttt{2D}} \delta(y) \delta(z),
\end{equation}
%
where $\delta$ represents the Dirac delta function. In this formulation, the superscript in $\boldsymbol{f}^i$ signifies the idealized case, where the body force is concentrated at the origin of the local reference frame (see Fig. ~\ref{fig:2dref}). 
%
When considering all cross-sections of a wing, the region of non-zero $\boldsymbol{f}^i$ forms a line in space, representing the theoretical limit of the actuator line method as the smearing parameter $\varepsilon$ approaches zero. Notably, while $\boldsymbol{f}^i$ is singular at the origin, its integral over the two-dimensional plane yields the expected value of $\boldsymbol{F}_{\texttt{2D}}/\rho$, thereby preserving consistency with the original force definition.
%
In the ALM, to avoid numerical problems related to singularities, the forces need to be distributed smoothly on several mesh points. This is usually performed by convolving the force with a regularization kernel. A common choice for the regularization kernel is a three-dimensional Gaussian kernel.  Considering the same constant Gaussian width in the three directions:
\begin{equation}
\label{eq:eta3d}
\eta_3(x,y,z) \defeq \frac{1}{\pi^{3/2}\varepsilon^3} \exp{\left(-\frac{x^2+y^2+z^2}{\varepsilon^2}\right)}=\eta(x)\eta(y)\eta(z)
\end{equation}
where
\begin{equation}
\eta(x) \defeq \frac{1}{\pi^{1/2}\varepsilon} \exp{\left(-\frac{x^2}{\varepsilon^2}\right)}
\end{equation}
The parameter $\varepsilon$ represents the projection width, defined as the maximum value among the chord factor, drag factor, and mesh factor, expressed as:
\[
\varepsilon = \max \left[\frac{c}{4}, \frac{c C_{d}}{2}, 4 \sqrt[3]{\mathcal{V}_{\text{cell}}}\right],
\]
where $c$ denotes the chord length, $C_{d}$ is the drag coefficient, and $\mathcal{V}_{\text{cell}}$ represents the volume of a computational cell. The smeared force is
\begin{equation}
\boldsymbol{f} = \boldsymbol{f}^i * \eta_3 .
\label{eq:fconvoluted}
\end{equation}
%
The convolution operation in the actuator line method is presented here using an idealized body force represented by Dirac delta functions as an intermediate step. Although this approach may initially appear unconventional, it is mathematically equivalent to the classical formulations~\citep{sorensen2002numerical, mikkelsen2003actuator}, which rely on one-dimensional integrals. This formalism, previously employed by \cite{martinez2019filtered}, establishes a rigorous connection between the two-dimensional force distribution and the three-dimensional convolution process.
%
To account for tip effects and rotational influences, a tip correction function is defined as follows~\citep{shen2012actuator}:
\[
f_{\text{tip}} = \frac{2}{\pi} \arccos \left[\exp \left(-g \frac{B(R-r)}{2 r \sin (\alpha+\beta)}\right)\right],
\]
where $\beta$ represents the pitch angle, encompassing both blade pitch and twist. The coefficient $g$ depends on various parameters, including the number of blades, the tip-speed ratio (TSR, denoted as $\lambda$), and the chord distribution. For simplicity, $g$ is expressed as a function of $B\lambda$:
\[
g = \exp \left[-C_1\left(B \lambda - C_2\right)\right] + 0.1.
\]
The coefficients $C_1$ and $C_2$ were empirically determined using data from the NREL rotor at a wind speed of $10~\mathrm{m/s}$ \citep{shen2005tip}. Their respective values are $C_1 = 0.125$ and $C_2 = 21$. A small offset of $0.1$ is included to ensure consistency for cases with an infinite tip-speed ratio or an infinite number of blades. The function $g$ adjusts the influence of tip vortices on the pressure distribution near the blade tip region. While Shen's tip correction model is not a perfect physical representation for the ALM applications, it has demonstrated promising results for such implementations, as supported by several studies~\citep{shen2014study,yu2018study,papi2021development}.


\bibliographystyle{abbrvnat}

\bibliography{references}


\end{document}

